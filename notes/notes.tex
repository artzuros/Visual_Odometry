\documentclass{article}
\usepackage{amsmath, amssymb, graphicx}

\title{Understanding Feature Matching, Essential Matrix, and Epipolar Geometry}
% \author{Your Name}
% \date{\today}

\begin{document}

\maketitle

\section{Introduction}
% This document explains the step-by-step process of:/
\begin{itemize}
    \item Finding feature correspondences using SIFT and ORB
    \item Computing the fundamental and essential matrices
    \item Decomposing the essential matrix to find camera motion
    \item Visualizing epipolar lines
\end{itemize}

\section{Feature Matching: Finding Correspondences}
To estimate the relationship between two images, we first need to find corresponding points between them.

\subsection{SIFT (Scale-Invariant Feature Transform)}
SIFT is a feature detection algorithm that finds keypoints in an image based on differences in scale and orientation.

\begin{enumerate}
    \item Convert the image to grayscale.
    \item Generate a scale space by progressively blurring the image.
    \item Use the Difference of Gaussians (DoG) to detect keypoints.
    \item Assign an orientation to each keypoint.
    \item Compute a descriptor by considering gradient orientations around the keypoint.
    \item Match keypoints between images using the Euclidean distance of their descriptors.
\end{enumerate}

The descriptor for a keypoint is represented as a 128-dimensional vector.

\subsection{ORB (Oriented FAST and Rotated BRIEF)}
ORB is a faster alternative to SIFT and is based on binary descriptors.

\begin{enumerate}
    \item Detect keypoints using the FAST corner detector.
    \item Assign an orientation using the intensity centroid method.
    \item Compute binary descriptors using the BRIEF (Binary Robust Independent Elementary Features) method.
    \item Match features using the Hamming distance instead of Euclidean distance.
\end{enumerate}

\section{Fundamental Matrix: Finding the Relationship Between Two Views}
The fundamental matrix \( F \) represents the relationship between two images by encoding how points in one image correspond to points in the other.

\subsection{Mathematical Definition}
The fundamental matrix satisfies the equation:

\begin{equation}
    \mathbf{x'}^T F \mathbf{x} = 0
\end{equation}

where:
\begin{itemize}
    \item \( \mathbf{x} \) is a point in the first image.
    \item \( \mathbf{x'} \) is the corresponding point in the second image.
    \item \( F \) is a \( 3 \times 3 \) matrix that encodes the relationship.
\end{itemize}

\subsection{Computing the Fundamental Matrix}
To compute \( F \), we use the Eight-Point Algorithm:

\begin{enumerate}
    \item Normalize the point coordinates.
    \item Construct a matrix \( A \) using the correspondences:
    \begin{equation}
        A = 
        \begin{bmatrix}
            x_1' x_1 & x_1' y_1 & x_1' & y_1' x_1 & y_1' y_1 & y_1' & x_1 & y_1 & 1 \\
            x_2' x_2 & x_2' y_2 & x_2' & y_2' x_2 & y_2' y_2 & y_2' & x_2 & y_2 & 1 \\
            \vdots & \vdots & \vdots & \vdots & \vdots & \vdots & \vdots & \vdots & \vdots \\
        \end{bmatrix}
    \end{equation}
    \item Compute the singular value decomposition (SVD) of \( A \), keeping only the smallest singular value.
    \item Ensure \( F \) has rank 2 by forcing the smallest singular value to be zero.
\end{enumerate}

\section{Essential Matrix: Incorporating Camera Calibration}
The essential matrix \( E \) is similar to the fundamental matrix but incorporates the camera’s intrinsic parameters.

\begin{equation}
    E = K^T F K
\end{equation}

where \( K \) is the camera intrinsic matrix containing focal length and optical center.

\subsection{Computing the Essential Matrix}
To compute \( E \):

\begin{enumerate}
    \item Compute the fundamental matrix \( F \).
    \item Multiply by the intrinsic camera matrix \( K \):
    \begin{equation}
        E = K^T F K
    \end{equation}
    \item Ensure \( E \) has rank 2 by forcing one singular value to be zero.
\end{enumerate}

\section{Decomposing the Essential Matrix: Extracting Rotation and Translation}
The essential matrix can be decomposed to extract the rotation and translation between two camera views.

\subsection{Using Singular Value Decomposition (SVD)}
We perform SVD on \( E \):

\begin{equation}
    E = U \Sigma V^T
\end{equation}

where \( U \) and \( V \) are orthogonal matrices and \( \Sigma \) contains singular values.

\subsection{Extracting Rotation and Translation}
From SVD, the two possible rotation matrices are:

\begin{equation}
    R_1 = U W V^T, \quad R_2 = U W^T V^T
\end{equation}

where:

\begin{equation}
    W = \begin{bmatrix} 0 & -1 & 0 \\ 1 & 0 & 0 \\ 0 & 0 & 1 \end{bmatrix}
\end{equation}

The translation vector is extracted as:

\begin{equation}
    t = U[:, 3]
\end{equation}

\section{Epipolar Geometry: Visualizing the Correspondence}
\subsection{Epipolar Lines}
Once we have \( F \), we can draw epipolar lines, which are the possible locations where a point in one image appears in the other.

\begin{equation}
    \text{epipolar line} = F \mathbf{x}
\end{equation}

where \( \mathbf{x} \) is a point in one image.

\section{Conclusion}
In this document, we covered:
\begin{itemize}
    \item How to find corresponding points using SIFT and ORB
    \item How to compute the fundamental and essential matrices
    \item How to extract rotation and translation from the essential matrix
    \item How to visualize epipolar geometry
\end{itemize}

These concepts are fundamental for 3D reconstruction, visual odometry, and structure-from-motion tasks.

\end{document}
